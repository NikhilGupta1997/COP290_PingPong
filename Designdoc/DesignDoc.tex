\documentclass{article}
\usepackage[utf8]{inputenc}
\usepackage{listings}
\usepackage{color} 
\usepackage{hyperref}
\usepackage{float}
\usepackage{subcaption}

\hypersetup{
    colorlinks=true, %set true if you want colored links
    linktoc=all,     %set to all if you want both sections and subsections linked
    linkcolor=blue,  %choose some color if you want links to stand out
}

%\usepackage{titling}
\usepackage{graphicx}
%\usepackage{titlepic}

\lstset{
	frame=tb, % draw a frame at the top and bottom of the code block
   	tabsize=4, % tab space width
   	showstringspaces=false, % don't mark spaces in strings
    numbers=left, % display line numbers on the left
    commentstyle=\color{red}, % comment color
    keywordstyle=\color{blue}, % keyword color
    stringstyle=\color{green} % string color
}


\title{\vspace*{\fill} \textbf{COP 290 Assignment 3}
	  \\ {\Large \textbf{Ping Pong Game}}
	  % \\  \vspace{3mm} \includegraphics{ddlogo.png}}
}
\author{
	\vspace{5mm} \includegraphics[width=5cm]{logo.png} \\
	 \textbf{Aditi}\\
	2014CS10205 \vspace{2mm} \\
	\textbf{Ayush Bhardwaj}\\ 
	2014CS10091 \vspace{2mm} \\
	\textbf{Nikhil Gupta}\\ 
	2014CS50462 \vspace{2mm}
}
\date{\vspace{3mm} \textbf{April 2016} \vspace*{\fill}}

\begin{document}
	\maketitle

	\newpage

	\tableofcontents

	\newpage

	\section{Objectives}
	Design a desktop app which is:

	\section{Overall Design}
		%The game which we will build is space invaders. It involves the player controlling a space ship and shooting down aliens. The aliens will fight back with bullets and missiles. The player has a limited number of lives and has to score the maximum in them.
				\begin{enumerate}
			\item The server side will be programmed in Web2py ~\cite{Web2py_Basics} 
			\item Volley will be used to send requests and receive responses
			\item Doxygen will be used to create HTML documentation of the entire code base.
			\item The entire code will be split up in multiple files to ensure modularity in code. 
		\end{enumerate}
	\section{User Interface}
	

    	\subsection{Front End}
    	% TODO: all kinds of random floating objects. NIKHIL do this.


	\section{Sub Components}
			\subsection{Physics}
			Corner cases :  corners 
					%\subsubsection
					\subsubsection{Databases}
					\begin{table}[H]
\centering
\caption{User Database Table}
\label{my-label}
\begin{tabular}{|c|c|c|c|}
\hline
\textbf{S.No.} & \textbf{Fields} & \textbf{Type} & \textbf{Description}                                             \\ \hline
1              & Name            & String        & Name of the person                                               \\ \hline
2              & Unique Id       & String        & Entry Number for students or Employee Code for staff and faculty \\ \hline
3              & User Type       & Int           & Information regarding user being a student or staff or faculty   \\ \hline
4              & Contact Number  & String        & Phone Number                                                     \\ \hline
5              & Hostel          & String        & Hostel if any                                                    \\ \hline
6              & Other Details   & String        & Any other details                                                \\ \hline
7              & Password	 	 & String        & Password in hashed form                                          \\ \hline
8              & User Id                    & String        & Entry Number for students or Employee Code \\ \hline
9              & Hostel Preferences         & String        & Bit string to represent interest in Hostel activities            \\ \hline
10              & Institute Preferences & String        & Bit string to represent interest in Institute activities         \\ \hline
11             & Extra Preferences          & String        & Bit String to represent interest in Other activities             \\ \hline

\end{tabular}
\end{table}




\begin{table}[H]
\centering
\caption{Hostel Level Complaint}
\label{my-label}
\begin{tabular}{|c|c|c|c|}
\hline
\textbf{S.No.} & \textbf{Fields}     & \textbf{Type} & \textbf{Description}                            \\ \hline
1              & Complaint Id        & String        & Unique Id for Complaint                         \\ \hline
2              & User Id             & String        & Unique User Id                    \\ \hline
3              & Complaint Type      & Int           & Complaint category                              \\ \hline
4              & Complaint Content   & String        & Content of complaint                            \\ \hline
5              & Extra Info          & Image         & Upload a photo                                  \\ \hline
6              & Admin Id            & String        & Id of person assigned                           \\ \hline
7              & Time Stamp          & Time          & Time of filing the complaint                    \\ \hline
8              & Resolved            & Boolean       & Resolved or Not                                 \\ \hline
9              & Mark for resolution & Boolean       & Option for complaint addressee to seek approval \\ \hline
10             & Comment             & String        & Any comments                                    \\ \hline
11             & Previous Id         & Int           & Previous complaint id if any                    \\ \hline
12             & Hostel              & Int           & Hostel Id                                       \\ \hline
13			& Anonymous 		& Boolean &  Anonymous or not \\ \hline
14              & Number of Up-votes      & Int           & Number of Up-votes                           \\ \hline
15             & Number of Down-votes    & Int           & Number of Down-votes                         \\ \hline
16              & Number of Neutrals     & Int           & Number of Neutral people                    \\ \hline
17              & Number of Satisfied    & Int           & Number of people satisfied                  \\ \hline
18              & Number of Dissatisfied & Int           & Number of people dissatisfied with solution \\ \hline

\end{tabular}
\end{table}


			\subsection{Computer Player Algorithm}
			\par\noindent The computer player can have variable speed. But the range of speed varies on the basis of level of difficulty.
			% make subsubsections for each event listed below!! TODO

			\begin{itemize}
			\item Initially computer player tries to align its paddle center with the center of the ball.
			\item When the ball is about to reach a wall, the computer player first calculates according to the assumption of a static paddle. And moves
			\item After the collision has occurred, the computer again moves accordingly.
			\item Another event will be choosing between ball catching versus catching power up objects. The decision will basically be based upon the distance of the power up object and the expected distance of the ball from the paddle after collision.

			% TODO: multiple balls case.
			\end{itemize}

			\subsection{p2p Networking}
			\par\noindent We would follow BLAH model of p2p networking. Each client sends the following information to its peers:
				\begin{enumerate}
					\item Crontab will also be used to generate push notifications for devices.
					\item In case a particular client is disconnected from the network, a pop up would be displayed to the other players, and according to their decision, a computer player will be added or the game would continue with one less player.
				\end{enumerate}
			

	\section{Interaction amongst Sub Components}
	
	\section{Testing Of Components}
		% \begin{enumerate}
			\subsection{Single peer testing}
				\begin{itemize}
					\item Unit testing will be used to check if the server and end points are working correctly.
					\item For each endpoint, stress testing will be done via python or bash scripts to verify that the APIs perform as expected in various situations.
				\end{itemize}
			\subsection{Overall p2p Testing}
				We will use the app on our desktops once it is ready to identify and squash any remaining bugs.
	\section{Extra Features}
	\begin{itemize}
	\item Every player gets an option of choosing the orientation of the board visible to him, according to his comfort.
	\item Special items would be floating around on the board, which on being captured, will provide the player with special powers
	\item The special powers include : extra speed, more length of paddle, extra points, SPIN?, 
	\item Multiple levels of the game by : ball speed, smaller paddle.
	\item While considering collision, frictional drag will also be considered.
	\item Corners of the board and the paddles will be circular.
	\item More than one balls can be 
	\end{itemize}

	\section{Future Endeavors} 

	\section{Source Code}
	The source code of the project is maintained in the following repository: \\
	\url{https://github.com/aditi741997/COP290_PingPong.git}
	\bibliographystyle{abbrv}
	\medskip
	\bibliography{references}
\end{document} 
